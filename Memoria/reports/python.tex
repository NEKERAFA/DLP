\section{Python's Implementation} % Implementación en Python

\hspace{2em}Interpreter used: \texttt{python3} version 3.5.2.\\

Operating System: Ubuntu Mate 16.04 (amd64).

\subsection{Language Survey} % Descripción

Python es un lenguaje de programación de propósito general multiparadigma. Es un lenguaje de alto nivel e interpretado, que nació con la filosofía de enfatizar la legibilidad del código y una sintaxis sencilla y compacta para que los desarrolladores programaran con la menor cantidad de lineas de código posible en comparación a otros lenguajes como C++ o Java. \\

Python usa un tipado fuerte y dinámico y no declarativo, permitiendo duck typing. Contiene un recolector de memoria automática, siendo el intérprete el encargado de reservar y eliminar la memoria según procesa el código. \\

La implementación de referencia del intérprete de Python es CPython, el cual es open source y es administrado por la Python Software Foundation.

\subsection{Analysis}

\subsubsection{Advantages and Disadvantages} % Análisis de ventajas y desventajas

\subsubsection{Implementation} % Análisis de implementacitón

\subsubsection{Possible Enhancements} % Posibles mejoras
