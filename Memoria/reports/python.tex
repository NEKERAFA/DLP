\section{Python's Implementation} % Implementación en Python

\hspace{2em}Interpreter used: \texttt{python3} version 3.5.2.\\

Operating System: Ubuntu Mate 16.04 (amd64).

\subsection{Language Survey} % Descripción

Python es un lenguaje de programación de propósito general multiparadigma. Es un lenguaje de alto nivel e interpretado, que nació con la filosofía de enfatizar la legibilidad del código y una sintaxis sencilla y compacta para que los desarrolladores programaran con la menor cantidad de lineas de código posible en comparación a otros lenguajes como C++ o Java. \\

Python usa un tipado fuerte y dinámico y no declarativo, permitiendo duck typing. Contiene un recolector de memoria automática, siendo el intérprete el encargado de reservar y eliminar la memoria según procesa el código. \\

La implementación de referencia del intérprete de Python es CPython, el cual es open source y es administrado por la Python Software Foundation.

\subsection{Analysis}

\subsubsection{Advantages and Disadvantages} % Análisis de ventajas y desventajas

Lo primero a comentar es que en Python no tenemos gestión manual de memoria, por lo que no existen los punteros y la reserva y liberación de memoria manual como ocurre en Java y OCaml. \\

Por otro lado, los parámetros de las funciones son pasados por referencia en Python. Esto nos ha permitido simplificar la forma de trabajar con el árbol, pareciéndose a la implementación de Pascal en muchos casos.

\subsubsection{Implementation} % Análisis de implementación

La principal diferencia en la implementación con respecto a todos los demás lenguajes analizados ha sido la traducción de tipos. En Python las variables son de tipo dinámico y no se declaran, por lo que se ha eliminado toda la definición de tipos presente en los demás lenguajes. \\

% TODO comprobar el funcionamiento de char y strings

% Al no haber tipos, no hay inguna forma de forzar al programador de que todas las claves sean del mismo tipo, por lo que puede ser inseguro

\subsubsection{Possible Enhancements} % Posibles mejoras
