\section{Implementación en C}
\hspace{2em}Para las pruebas en este lenguaje se usó el compilador \texttt{gcc} en su versión 5.4.0 sobre Ubuntu Mate 16.04 (amd64).

\subsection{Descripción}
\hspace{2em}C es un lenguaje de programación de propósito general que sigue el paradigma imperativo procedimental estructurado. Es un lenguaje de nivel medio ya que permite el uso de procedimientos y estructuras de alto nivel así como incluir código ensamblador en el código fuente.\\

La compilación del código fuente se traduce a código máquina para la plataforma final. A pesar de esta restricción, como su principal uso fue dirigido a la creación de sistemas operativos, a día de hoy es un lenguaje multiplataforma debido a que sus compiladores permiten compilarlo para cualquier plataforma final sin hacer apenas cambios en el código. \\

Su sintaxis viene especificada en el estándar C, definiéndolo como lenguaje de tipado estático, débil, nominal declarativo. Es el programador el encargado de la gestión de memoria debido a que el lenguaje permite trabajar a bajo nivel para reservar memoria dinámica y liberarla, por lo que la gestión de memoria es manual. \\

La enorme popularidad del lenguaje le ha permitido ser el lenguaje que más implementaciones tiene. A día de hoy sirve de lenguaje intermedio para otros lenguajes de alto nivel como C++, C\#, Objetive-C, Go, Perl, PHP, Python, Lua y Swift. También contiene muchísimas bibliotecas externas para emular comportamientos de otros lenguajes, como la gestión de memoria automática, orientación a objetos mediante GObject e incluso poder usar funciones lamda.
\subsection{Análisis}

\subsection{Modificaciones en el diseño}
