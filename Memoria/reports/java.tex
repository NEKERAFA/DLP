\section{Implementación en Java}
\hspace{2em}Para este lenguaje se usó el compilador \texttt{javac} en su versión 1.8.0\_144. %Falta el SO

\subsection{Descripción}
\hspace{2em}Java es un lenguaje de programación de alto nivel que sigue el paradigma de orientación a objetos basada en clases, aunque a día de hoy las últimas versiones ya incluyen elementos de otros paradigmas, como funciones-lambda. \\

Como en sus inicios se pensó en que fuera un lenguaje de propósito general, la compilación del código fuente se traduce a bytecode java, el cual es ejecutado sobre una máquina virtual que sí está diseñada para la plataforma final. \\

Su sintaxis deriva en gran medida de C++, siendo Java un lenguaje de tipado estático y fuerte. Es la propia máquina virtual la encargada de la gestión de memoria mediante un recolector automático de basura, el cual es responsable de gestionar el ciclo de vida, reservando memoria cuando el programador crea un objeto, y liberando memoria cuando se deja de referenciar completamente el objeto usado.

\subsection{Análisis del código} % Quizás quedara mejor poner primero imperativo ya que se supone que es la clase más general
\hspace{2em}Dadas las diferencias entre Pascal (lenguaje estructurado imperativo) y Java (lenguaje orientado a objetos imperativo), la estructura \texttt{tNodoA} se ha traducido en la clase \texttt{Node}, que sigue conteniendo los atributos \texttt{key}, \texttt{rightChild} y \texttt{leftChild}. Como las clases son más abstactas que las estructuras, la programación resultante es de alto nivel y más legible. \\

Otro elemento que ha cambiado son los punteros y la gestión de memoria manual en Pascal. En Java no hay elementos para manipulación de bajo nivel, por lo que la gestión de memoria es automatizada por la máquina virtual. Esto hace que desaparezca el uso de los punteros \texttt{tPosA}. \\

Relacionado, se ha eliminado el procedure \texttt{error}, ya que la máquina virtual de Java, cuando se queda sin memoria, lanza la excepción \texttt{OutOfMemoryError}. También se ha eliminado la función \texttt{crearNodo}, que se traduce en el constructor de la clase \texttt{Node} en Java.

%%%%%%%%%%%%%%%%%%%%%% Análisis de ventajas y desventajas %%%%%%%%%%%%%%%%%%%%%%%%%%%%%%%%%%%%
Como se ha mencionado anteriormente, Java no permite la gestión manual de la memoria, por lo que los tipos de datos que eran punteros no se han podido "traducir" directamente. La gestión de memoria automática nos permite olvidarnos de las tareas de reservar espacio, gestionar la falta de memoria (ya que la propia máquina virtual de Java se encarga de lanzar la excepción \texttt{OutOfMemoryError}) cuando esto sucede) y liberar la memoria, posibilitando que nos centremos más en los algoritmos en sí. \\

Por otro lado nos encontramos con que en Java el paso de parámetros es siempre por valor y no por referencia, lo que ha ocasionado problemas que no existían en Pascal a la hora de manejar la estructura del árbol (explicaremos estos problemas más adelante cuando entremos en detalles de la implementación). \\

Por último, cabe destacar que al utilizar clases y no estructuras y punteros, el código resultante es de más alto nivel y más legible. \\

% Comentar que se puede hacer árbol genérico, poner los métodos en el árbol, atributos privados...

%%%%%%%%%%%%%%%%%%%%%% Análisis de implementación %%%%%%%%%%%%%%%%%%%%%%%%%%%%%%%%%%%%%%%%%%%%%
En primer lugar hablaremos de la "traducción" de los tipos. 
\begin{itemize}
	\item \texttt{tPosA}, al ser un puntero, desaparece.
	\item \texttt{tNodoA} se convierte en la clase \texttt{Node}, con los mismos atributos que tenía como campos la estructura original. Siguiendo los principios de la orientación a objetos, los atributos son privados, pero tenemos getters públicos. Por otro lado, los setters tienen visibilidad paquete para que puedan ser usados desde la clase Bst, pero no desde el programa principal.
	\item \texttt{tABB} se convierte en la clase Bst\footnote{Binary Search Tree}, que tiene un atributo \texttt{Node} representando la raíz del árbol. De este modo, Bst actúa como un envoltorio y punto de acceso a la estructura de nodos. Todas las funciones y procedimientos definidos en el TAD de Pascal son métodos de esta clase.
\end{itemize}
Volviendo sobre los problemas de paso de parámetros mencionados anteriormente, tenemos que, en concreto, cuando un objeto se pasa como argumento, lo que se recibe es una "copia" del puntero a dicho objeto, por lo que todos los cambios realizados en ese puntero no se reflejan fuera del método. Es por esto que ha resultado necesario recurrir a "punteros indirectos" como utilizar los nodos padre para manejar a los hijos, o utilizar el objeto Bst que envuelve la raíz para insertar y eliminar la misma.\\ % Explicar después


% Decisión respecto a la clave Integer cuando esté tomada

Ahora hablemos de la adaptación de los procedimientos y funciones.
\begin{itemize}
% Decisión respecto a arbolVacio cuando esté tomada
	\item \texttt{hijoIzquierdo}, \texttt{hijoDerecho} y \texttt{raiz}: Devuelven los atributos correspondientes del nodo alojado en la raíz del árbol. En el caso de los dos primeros, se devuelven envueltos en un nuevo objeto Bst para seguir la interfaz declarada en Pascal en cuanto al tipo devuelto.
	\item \texttt{esArbolVacio}: Ya que para que el método pueda ser llamado el propio árbol no puede ser nulo, se comprueba si % Decisión respecto al criterio de árbol vacío cuando esté tomada
	\item \texttt{insertarClave}: % Si tomamos que en árbol vacío root == null, al insertar en árbol vacío sí que hay que crear un nodo, lo que cambiaría la implementación de insertar iterativo y la haría más acorde a la implementación de Pascal
	\item \texttt{buscarClave}: Al igual que con \texttt{hijoIzquierdo} y \texttt{hijoDerecho}, el nodo a devolver se envuelve en un objeto Bst para cumplir con la interfaz de Pascal.
	\item \texttt{eliminarClave}: Tanto la adaptación de la versión recursiva de este procedimiento como del procedimiento auxiliar \texttt{sup2} reciben ahora un argumento más: el padre del nodo con mayor clave del subárbol izquierdo, necesario para poder reorganizar el árbol, ya que el paso por valor de Java no permite hacerlo de otra forma. % Explicar infinitamente mejor cuando no tenga tanta hambre
	% Decisión respecto a si función lambda o no cuando esté tomada
\end{itemize}
Además cabe destacar que las versiones iterativas de las funciones necesitan menos argumentos que su versión en Pascal, ya que no necesitan recibir el árbol a tratar.


\subsection{Modificaciones en el diseño}

Por aquí poner todo lo que ha cambiado en la implementación.