\section{Implementación en Java}

Para este lenguaje se usó el compilador \texttt{javac} en su versión 1.8.0\_144.

\subsection{Descripción}

Java es un lenguaje de programación de alto nivel que sigue el paradigma de orientación a objetos basada en clases, aunque a día de hoy las últimas versiones ya incluyen elementos de otros paradigmas, como funciones-lamda. \\

Como en sus inicios se pensó en que fuera un lenguaje de propósito general, la compilación del código fuente se traduce a bytecode java, el cual es ejecutado sobre una máquina virtual que si está diseñada para la plataforma final. \\

Su sintaxis deriva en gran medida de C++, siendo Java un lenguaje de tipado estático y fuerte. Es la propia máquina virtual la encargada de la gestión de memoria mediante un recolector automático de basura, el cual es responsable de gestionar el ciclo de vida, reservando memoria cuando el programador crea un objeto, y liberando memoria cuando se deja de referenciar completamente el objeto usado. \\

\subsection{Análisis}

\subsection{Justificaciones de diseño}
